\documentclass[12pt, letterpaper, titlepage]{article}
\usepackage[left=3.5cm, right=2.5cm, top=2.5cm, bottom=2.5cm]{geometry}
\usepackage[MeX]{polski}
\usepackage[utf8]{inputenc}
\usepackage{graphicx}
\usepackage{enumerate}
\usepackage{amsmath} %pakiet matematyczny
\usepackage{amssymb} %pakiet dodatkowych symboli
\title{Dokument tekstowy typu latex}
\author{Wiktor Łępicki}
\date{Październik 2k22}
\begin{document}
\maketitle
\begin{center}
\textbf{Artykuły z internetu}
\begin{enumerate} %komentarz
\item Punkt 1
\\Jak żyli neandertalczycy?
\item Point 2
\\Co stało się z neandertalczykami?
\item Punto 3
\\Pierwsza zidentyfikowana rodzina neandertalczyków.
\end{enumerate}
\end{center}
\newpage
\section{Jak żyli neadertalczycy}
\textbf{Czy neandertalczycy żyli w rodzinach?} Najnowsze odkrycie w syberyjskich jaskiniach rzuca nowe światło na codzienność i więzi społeczne naszych przodków.
\subsection{Co stało się z neandertalczykami?}
\textsf{Istnieje wiele teorii tłumaczących wyginięcie neandertalczyków. Jedna z nich mówi, że degradacja ich terytoriów nastąpiła na długo przed przybyciem współczesnego człowieka.
\\{\LARGE To doprowadziło do zdziesiątkowania i ostatecznego zaniku populacji.}
\\Duża część naukowców sugeruje, że różne reakcje na zmieniający się klimat i środowisko oraz wykorzystanie zasobów również mogły odegrać ważną rolę w zniknięciu neandertalczyków.}\subsubsection{Pierwsza zidentyfikowana rodzina neandertalczyków}
\textit{Pierwszy szkic genomu neandertalczyka został opracowany w 2010 r. Od tego czasu naukowcy z Instytutu Antropologii Ewolucyjnej im. Maxa Plancka zsekwencjonowali kolejnych 18 genomów z 14 różnych stanowisk archeologicznych w całej Eurazji. O poszczególnych społecznościach neandertalczyków nadal wiemy jednak niewiele.}

\end{document}


